
% ctan tug
\documentclass[a4paper,capchap,espacoduplo,normaltoc]{abntepusp}

%\usepackage[bookmarks,pdftex,a4paper,colorlinks=true,citecolor=black,urlcolor=blue,linkcolor=black,pdfpagemode=None]{hyperref}
%\usepackage[bookmarks,a4paper,colorlinks=true,citecolor=black,urlcolor=blue,linkcolor=black,pdfpagemode=None]{hyperref}
\usepackage[centertags]{amsmath}
\usepackage{amsfonts}
\usepackage{amssymb}
\usepackage{amsthm}
\usepackage[T1]{fontenc}
\usepackage[latin1]{inputenc}
\usepackage[brazil]{babel}
%\usepackage[alf,abnt-repeated-author-omit=yes]{abntcite}

\usepackage{url} % \url{http...}
\usepackage{helvet}\renewcommand{\familydefault}{\sfdefault}
%\usepackage{winfonts}
%\usepackage{txfonts}
%\usepackage{tabela-simbolos}

\fontfamily{arial}\selectfont
\renewcommand{\rmdefault}{arial}

% Math -------------------------------------------------------------------
\newtheorem{theorem}{Teorema}{\bfseries}{\itshape}
\newtheorem{lemma}{Lema}{\bfseries}{\itshape}
\newtheorem{definition}{Defini��o}{\bfseries}{\itshape}
\newtheorem{corollary}{Corol�rio}{\bfseries}{\itshape}

\sloppy

\begin{document}

\renewcommand{\bibname}{Refer\^encias} % adapta��o para ABNT/EPUSP

\autorPoli{Felipe G. Godoy, Pedro d'Aquino, Rafael Ruppel, Rafael da Silva}

\titulo{SAURON}

\orientador{Anna Reali Costa}

\monografiaFormatura
%\monografiaMBA
%\qualificacaoMSc{<�rea do Mestrado>}
%\qualificacaoMSc{Enge\-nharia El�trica}
%\dissertacao{<�rea do Mestrado>}
%\qualificacaoDr{<�rea do Mestrado>}
%\teseDr{<�rea do Doutorado>}
%\teseLD
%\memorialLD

%\areaConcentracao{<�rea de Concentra��o>}
\areaConcentracao{Sistemas Digitais}

%\departamento{<Departamento>}
\departamento{Departamento de Engenharia de Computa��o e Sistemas Digitais (PCS)}

\local{S�o Paulo}

\data{2009}

\dedicatoria{Dedico esse trabalho a meus pais.......}

\capa{}

\folhaderosto{}

%
% Ficha Catalogr�fica
% (deve ficar no verso da p�gina de rosto)
%
\renewcommand{\PoliFichaCatalograficaData}{%
  1. Assunto \#1. 2. Assunto \#2. 3. Assunto \#3.
  I. Universidade de S�o Paulo. Escola Polit�cnica.
  \PoliDepartamentoData. II. t.}

%\fichacatalografica % formata a ficha

\paginadedicatoria{}

\begin{agradecimentos}
Vou citar~\cite{agashe-lauter-venkatesan}
\end{agradecimentos}

\begin{resumo}
\end{resumo}

\begin{abstract}
\end{abstract}

\begin{resume}
\end{resume}

\begin{zusammenfassung}
\end{zusammenfassung}

\tableofcontents

\listoffigures

\listoftables

\chapter{Introdu��o}\label{chp:intro}

\section{Apresenta��o}\label{sec:presentation}

\section{Objetivos}\label{sec:goals}

\section{Contribui��es originais}\label{sec:contrib}

\section{Organiza��o}

\chapter{Outro cap�tulo}\label{chp:outrocap}

Este cap�tulo desenvolve a teoria de~\cite{agashe-lauter-venkatesan,al-riyami-paterson,ansi-x9.62,balasubramanian-koblitz,blake-seroussi-smart,bleichenbacher,weimerskirch}.

\[
E = mc^2.
\]

\chapter{Conclus�es}\label{chp:conclusion}
\bibliographystyle{alpha}
\bibliography{biblio}

\appendix

\chapter{Ap�ndice}\label{app:apendiceA}

\end{document}