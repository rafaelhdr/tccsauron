\section{Materiais e m�todos}\label{sec:materiaisMetodos}



De acordo com o que foi dito na Revis�o Liter�ria \ref{sec:revisaoLiteratura}, este trabalho segue arquitetura semelhante � utilizada em \cite{barra}.
Assim, dentro desta se��o ser� detalhada a Arquitetura Geral do Sistema \ref{subsec:arquitetura} e, em seguida, ser�o tratados os m�dulos componentes dessa arquitetura. Come�ando pelos dois grandes m�dulos constituintes dessa arquitetura: Navega��o \ref{subsec:navegacao} e Localiza��o \ref{subsec:localizacao}. A seguir, os componentes do M�dulo de Localiza��o ser�o detalhados: o Modelo de Din�mica \ref{subsec:modelodinamica} e os Modelos de Observa��o: Modelo de Observa��o dos Sonares \ref{subsec:modelosonares}, que se refere � nossa primeira abordagem; e Modelo Simples de Observa��o dos Sonares \ref{subsec:modeloSimplesSonar}, que se refere ao modelo final do trabalho e, por �ltimo, o Modelo de Observa��o da Vis�o \ref{subsec:modeloVisao}


\documentclass[a4paper,11pt]{article}
\usepackage{graphicx}
\usepackage[brazilian]{babel}
\usepackage[latin1]{inputenc}
\usepackage[T1]{fontenc}
\usepackage{amsmath, amsthm, amssymb, bm}
\usepackage[ruled,vlined,portugues]{algorithm2e}



\numberwithin{equation}{section}
\numberwithin{figure}{section}

\begin{document}



\subsection{Especifica��o}

O SAURON possui as seguintes especifica��es de Hardware e Software:

\begin{itemize}

\item{Plataforma rob� Pioneer 2DX, contendo:

\begin{itemize}
\item{Estrutura Mec�nica: motores e rodas;}
\item{Od�metro}
\item{Conjunto de 8 sonares;}
\end{itemize}}

\item{Notebook Embarcado;}
\item{C�mera de baixo custo;}
\item{Conex�o Serial-USB entre a plataforma-rob� e o notebook;}
\item{Conex�o USB entre c�mera e notebook ;}
\item{Software de Localiza��o;}
\item{Software de Navega��o;}


\item{Notebook }


\end{itemize}

\subsection{Arquitetura}\label{subsec:arquitetura}

Aqui � arquitetura geral. Planejamento+Localiza��o+GUI+Sensores
\subsection{Navega��o}\label{subsec:navegacao}

O m�dulo de Navega��o n�o estava inicialmente previsto no projeto. A inten��o era utilizar um software de navega��o que acompanha a plataforma rob� Pioneer 2DX, chamado SonARNL. Por�m, ao longo dos testes, percebeu-se que o comportamento do SonarNL atrapalhava o m�dulo de Localiza��o, e, portanto, decidiu-se por implementar um m�dulo de navega��o simplista, mas sobre o qual houvesse total controle.

O comportamento n�o desejado apresentado pelo SonarNL era o de realizar constantes mudan�as no \theta do rob�. � prov�vel que esse comportamento tivesse a inten��o de melhorar a precis�o do m�dulo de localiza��o, atrav�s da identifica��o de novos pontos de interesse para os sonares. Por�m, de fato, isso gerava altas incertezas, prejudicando o desempenho da localiza��o.


\subsubsection{Algoritmo}
O m�dulo de Navega��o usa o algoritmo de busca em grafo conhecido por A* (l�-se "A estrela"). Ao adaptar esse problema � navega��o, os n�s do grafo passam a representar pontos em um mapa. Simplificadamente, pode-se dizer que o algoritmo parte de um n� inicial do seu grafo e come�a a visitar os n�s adjacentes, usando um crit�rio de custo para determina��o de qual n� deve ser visitado. O custo considerado � o custo de se locomover at� o n� adjacente somado do custo de se locomover do n� adjacente at� o n� destino. 

Para possibilitar essa navega��o de busca em grafos, foram inseridos pontos de rota (ou \textit{waypoints}) no mapa utilizado pelo SAURON. Assim, os n�s do grafo utilizado pelo algoritmo s�o os \textit{waypoints} e os objetivos (ou \textit{goals}). Os \textit{goals} representam os pontos de interesse para os usu�rios (e.g., "sala C2-50", sala "D1-30", e etc.) e j� existiam no mapa originalmente. Os \textit{waypoints} s�o pontos artificiais, criados para servir de rota para a navega��o.

Os \textit{waypoints} s�o colocados afastados das bordas do mapa e em pontos estrat�gicos escolhidos para permitir a realiza��o de curvas e facilitar a locomo��o a qualquer um dos objetivos.

Apesar do grafo A* conter esses dois tipos de n�s: n� inicial, \textit{waypoints} e \textit{goals}, apenas os \textit{waypoints} s�o usados para estabelecer a rota. O \textit{goal} s� servir� como n� final, ou seja, destino da navega��o. O n� inicial � um n� tempor�rio, inexistente no mapa, criado no in�cio da execu��o do algoritmo com as coordenadas atuais do rob�. 

Com a estrutura do grafo pronta, basta aplicar o algoritmo A* para obter a sua rota.

\subsubsection{Controle de Retrocessos} 

Um problema que logo se manifestou para a navega��o era o cuidado especial com retrocessos. Algumas vezes, � necess�rio  afastar-se um pouco do destino antes de poder alcan��-lo.  A figura abaixo ilustra essa situa��o. Para ir do ponto A ao ponto D, deve-se contornar a parede, passando por exemplo, pelo ponto B, que est� mais distante do destino D do que o pr�prio A.

\missingfigure{Criar um mapazinho onde o texto acima � verdade, colocar distancia pontos e distancia rota}

Para resolver esse problema basta considerar o custo de um n� at� o destino como sendo o custo da rota e n�o como a dist�ncia cartesiana entre esses dois pontos. Al�m disso � preciso determinar que os n�s adjacentes s� poder�o ser n�s alcan��veis, ou seja, n�s entre os quais � poss�vel tra�ar uma linha reta, sem nenhum obst�culo entre eles. De fato, foi determinado que um n� poderia ter, no m�ximo, 4 n�s adjacentes. As quatro possibilidades de adjac�ncia s�o: cima, baixo, esquerda e direita.

O problema de retrocessos, entretanto, ainda poderia existir. O m�dulo de navega��o recalcula sua rota o tempo todo, e como foi visto anteriormente, uma das primeiras etapas � a cria��o de um n� tempor�rio, para representar a posi��o inicial do rob�. O problema � que esse n� tempor�rio n�o possui adjac�ncia bem definida no mapa, ent�o � necess�rio buscar os n�s adjacentes desse n� tempor�rio inicial.

Definiu-se que o n� tempor�rio inicial possuiria um �nico n� adjacente. A busca desse n� adjacente � feita entre todos os n�s alcan��veis pelo n� inicial. A partir disso, o n� adjacente ser� aquele cujo custo de rota at� o destino seja o menor.

\subsubsection{Execu��o e Controle de Rota} 

Assim que definida uma rota de waypoints at� o destino final, passa-se ao Executor de Rota, que opera da seguinte forma:

\begin{itemize}
	\item Realiza um giro at� ficar direcionado ao pr�ximo waypoint.
	\item Desloca-se em linha reta at� alcan�ar o pr�ximo waypoint.
\end{itemize} 

Essa estrat�gia de execu��o visa minimizar as incertezas em \theta, realizando o menor n�mero poss�vel de curvas, que foi a grande motiva��o para a realiza��o deste m�dulo de Navega��o.

Al�m disso, o Executor de Rota tamb�m possui um comportamento de Evitar Colis�es, ou seja, o rob� interrompe seu deslocamento caso detecte um obst�culo � frente.

O m�dulo de navega��o tamb�m possui um Controle de Rota, para determinar quando ele se afastou da rota prevista. Quando isso ocorre, um callback � enviado para rec�lculo da rota.

\subsubsection{Navega��o InterMapas}

Como um adendo final ao projeto, adicionou-se uma camada para permitir a navega��o entre M�ltiplos Mapas, permitindo, por exemplo, subir uma rampa em dire��o a um outro andar de um pr�dio.

Para isso, carrega-se o sistema com todos os mapas dispon�veis e cria-se liga��es entre os mapas atrav�s de novos \textit{waypoints} chamados portais. De fato, basta que um waypoint em um mapa tenha o mesmo nome de outro waypoint em outro mapa para que seja estabelecido um portal entre eles.

Esta camada passou a ser o ponto de entrada do m�dulo de Navega��o e executa o algoritmo abaixo. A refer�ncia \textbf{pathPlanner} indica o algoritmo de Navega��o para um mapa simples.

   
 \begin{algorithm}
	\dontprintsemicolon
	
	\Entrada{destino}
	
	\Inicio{
		\Se{destino est� neste mapa}{\ pathPlaner vai at� destino \;}
		\Senao{
			descobre pr�ximo mapa na rota de mapas \;
			achar portal em rela��o ao mapa atual \;
			\Se{pathPlanner chega em portal}
			{\ realiza troca de mapas \;
  		\ recome�a  \;}
  		\Senao{
			 erro \;
		}
	}
	}
	\caption{Algoritmo de navega��o entre M�ltiplos Mapas}
		\label{algo:intermapas}
\end{algorithm}
\subsection{Localiza��o}\label{subsec:localizacao}

A localiza��o � o m�dulo mais complexo dentro da arquitetura deste trabalho. Ela foi projetada com base no filtro bayesiano EKF (\textit{Extended Kalman Filter}), melhor descrito no Ap�ndice A \ref{sec:apendicea}. Nesta se��o, encontra-se a vis�o geral dessa arquitetura e sua implementa��o. 

\subsubsection{Vis�o Geral}

O sistema de localiza��o estima a posi��o do rob� a partir dos dados de seus sensores. No SAURON, os sensores s�o os oito sonares e uma c�mera de v�deo. Cada sensor tem associado a si um modelo de observa��o, que descreve a rela��o da medida lida com o mundo real. Por exemplo, o modelo de observa��o do sonar associa uma s�rie de leituras a uma parede no mapa. De agora em diante, quando falarmos em um sensor, estamos nos referindo ao seu modelo de observa��o.

Cada sensor sabe qual � a observa��o esperada para uma dada postura do rob�. Uma diferen�a entre a observa��o esperada e a observa��o de fato recebida indica um erro na estimativa do sistema. Se esses erros forem pequenos, podem ser consequ�ncia das varia��es naturais dos sensores f�sicos. Contudo, diferen�as grandes devem ser utilizadas para corrigir a postura. O ponto mais importante do algoritmo de localiza��o � exatamente este: a partir da diferen�a entre as observa��es esperadas e reais, atualizar a estimativa da postura.

Essas estimativas, que s�o corrigidas pelos sensores, s�o previstas pelo modelo de din�mica. Seu papel � gerar uma estimativa a partir da �ltima postura do rob�, sabendo o quanto ele andou nesse intervalo. Existem duas abordagens a esse problema. A primeira � utilizar os sinais de controle, como velocidade de cada roda, para prever o movimento do rob�. A segunda, utilizada no SAURON, � obter as informa��es a partir do od�metro do rob�.

O filtro estendido de Kalman (EKF) � filtro matem�tico capaz de realizar a previs�o e corre��o das estimativas de postura. Por fim, o �ltimo componente do sistema � o m�dulo de Localiza��o, que coordena o funcionamento de todo o sistema, iniciando os sensores e ligando-os ao EKF.

A figura \ref{fig:arquitetura} ilustra a disposi��o dos m�dulos do SAURON. As informa��es retornadas pelos modelos dos sensores s�o a observa��o real ($z$), observa��o esperada ($h$), matriz de observa��o -- que descreve como a observa��o depende dos componentes $x$, $y$ e $\theta$ da postura estimada -- ($H$) e a covari�ncia ($R$).

\begin{figure}[ht]
	\centering
		\includegraphics[width=.8\columnwidth]{imagens/arquitetura.pdf}
	\caption{Arquitetura do sistema de localiza��o}
	\label{fig:arquitetura}
\end{figure}

\subsubsection{Implementa��o}

Uma dificuldade em se implementar um sistema como esse � o controle do fluxo de execu��o. Para explicar o problema, olhemos o algoritmo \ref{algo:localizacao}, um esbo�o do funcionamento b�sico do m�dulo de localiza��o.

\begin{algorithm}
\caption{Esbo�o do algoritmo do m�dulo de localiza��o}
\label{algo:localizacao}
\begin{algorithmic}
\LOOP
	\STATE $postura \gets ekf.predict(modeloDinamica.novaEsimativa())$
	\FORALL{$sensor$}
		\IF{$sensor.temMedidaNova()$}
			\STATE $postura \gets ekf.update(sensor.observacaoEsperada())$
		\ENDIF
	\ENDFOR
\ENDLOOP
\end{algorithmic}
\end{algorithm}

A implementa��o ing�nua desse algoritmo percorre os passos do pseudoc�digo sequencialmente, ou seja, itera por todos os sensores � procura daqueles que estejam prontos. A tabela \ref{tab:frequenciaSensores} mostra que os diferentes tempos de atualiza��o de cada sensor tornam essa abordagem invi�vel: na grande maioria das vezes, n�o haver� nenhum sensor pronto, e a �nica estimativa ser� aquela fornecida pelo modelo de din�mica. Al�m disso, o la�o de processamento � executado o tempo todo, consumindo CPU e, paradoxalmente, aumentando ainda mais o tempo de resposta dos sensores.

\begin{table}[htbp]
	\centering
		\begin{tabular}{ | l | r | }
		\hline
		Sensor & Tempo de atualiza��o (ms) \\ \hline
		Od�metro & 5 \\ \hline
		Vis�o & 25 \\ \hline
		Sonar & 100 \\ \hline			
		\end{tabular}
	\caption{Frequ�ncia de atualiza��o dos sensores e do od�metro}
	\label{tab:frequenciaSensores}
\end{table}

A solu��o � transferir o controle do fluxo de execu��o de um m�dulo central para os sensores. Quando um sensor receber novas medidas e for capaz de gerar uma observa��o esperada, ele chama o m�dulo de localiza��o, que ent�o invoca o EKF e computa a nova estimativa. Esse processo � muito mais eficiente do que o \textit{polling}, pois n�o h� recursos desperdi�ados enquanto se espera inutilmente. A desvantagem do m�todo distribu�do � que um certo tempo deve ser perdido realizando a sincroniza��o de todas as \textit{threads} (para evitar, por exemplo, que dois sensores atualizem a estimativa ao mesmo tempo, corrompendo-a). A tabela \ref{tab:usoProcessador} relata os resultados obtidos para cada abordagem.

\begin{table}[htbp]
	\centering
		\begin{tabular}{ | l | r | }
		\hline
		Abordagem & Utiliza��o m�dia dos processadores \\ \hline
		Centralizada & 50\% \\ \hline
		Distribu�da & 11\% \\ \hline
		\end{tabular}
	\caption{Uso do processador nas duas abordagens de implementa��o (sistema com duas CPUs)}
	\label{tab:usoProcessador}
\end{table}


\section{Modelo de Din�mica}

\subsection{Defini��o}
O Modelo de Din�mica � o respons�vel por fazer a predi��o no filtro de Kalman, utilizando para isso os valores do od�metro.

\subsection{Detec��o de Proje��es Verticais}
The quick brown fox jumps over the lazy dog.

\subsection{Perfil de Cor}
The quick brown fox jumps over the lazy dog.

\subsection{Rastreamento e Associa��o}
The quick brown fox jumps over the lazy dog.

\subsection{Pr�ximos Passos}
The quick brown fox jumps over the lazy dog.


\begin{comment}
\documentclass[a4paper,11pt]{article}
\usepackage{graphicx}
\usepackage[brazilian]{babel}
\usepackage[latin1]{inputenc}
\usepackage[T1]{fontenc}
\usepackage{fullpage}
\usepackage{amsmath, amsthm, amssymb, bm}
\usepackage[ruled,vlined,portugues]{algorithm2e}
\numberwithin{equation}{section}
\numberwithin{figure}{section}
\begin{document}
\end{comment}

\section{Modelos de Observa��o dos Sonares}

Um modelo � uma aproxima��o matem�tica do mundo real. No contexto de localiza��o rob�tica, modelos de observa��o s�o usados para representar o comportamento de sensores e, em especial, como as medidas obtidas se relacionam com grandezas do ambiente observado. 

Como descrito na se��o \ref{sec:localizacao}, a localiza��o bayesiana por filtro de Kalman utiliza as informa��es dos sensores para corrigir sua estimativa do estado atual. Essencialmente, a informa��o utilizada pelo filtro � a diferen�a entra a observa��o esperada (normalmente indicada por $h(\hat{x})$, onde $\hat{x}$ � a estimativa da postura) e a observa��o real (denotada por $z$). Por exemplo, suponha um rob� equipado com um sonar que incide perpendicularmente � parede, como na figura \ref{fig:exemplo_modelo}. Se ele estima estar em uma posi��o tal que a leitura esperada do sonar � 100 cm, mas a observa��o real � 80 cm, sabe-se que o rob� est� de fato mais perto da parede do que acredita. O papel do modelo de observa��o de um sensor � gerar as estimativas de observa��o $h(\hat{x})$.

\begin{figure}[b]
	\centering
		\includegraphics{imagens/sonar/exemplo_modelo.pdf}
			\caption{Um modelo de observa��o � usado para estimar o erro entre a postura real ($\vec{x}$) e a postura estimada ($\hat{x}$). Neste exemplo, um sonar perpendicular � parede recebe leitura $z$, mas o modelo de observa��o prev�, para a posi��o estimada $\hat{x}$, a leitura $h(\hat{x})$. Essa diferen�a ser� utilizada pelo filtro de Kalman para corrigir a estimativa da postura.}
	\label{fig:exemplo_modelo}
\end{figure}

Foram desenvolvidos dois modelos de observa��o dos sonares. O sonar � um sensor de profundidade que emite ondas sonoras e, a partir do tempo entre emiss�o e recep��o, calcula a dist�ncia ao obst�culo mais pr�ximo. Ele opera de maneira similar a um sensor laser, mas com menor precis�o e densidade. Contudo, um sensor laser � muito caro para algumas aplica��es. Um dos objetivos deste projeto � construir um rob� de baixo custo, utilizando sensores simples, de modo que a escolha do sonar se faz adequada.

\subsection{Modelo Baseado em Associa��es}
\label{sec:nossosonar}
A se��o \label{sec:sonarbarra} apresentou o modelo de observa��o desenvolvido em \cite{barra}. O modelo de observa��o basedado em associa��es � uma modifica��o desse modelo. Em particular, as equa��es foram refeitas para incluir o efeito do �ngulo relativo de cada sonar em rela��o ao centro do rob�, e a manuten��o de associa��es j� realizadas foi facilitada atrav�s do m�todo de rastreamento de paredes.

Este modelo � chamado de baseado em associa��es porque, antes de gerar observa��es esperadas e corrigir a postura estimada, � necess�ria uma confirma��o de que um sonar est� de fato observando uma determinada parede. A tarefa de relacionar as �ltimas leituras com uma parede do mapa chama-se associa��o. A principal motiva��o de um modelo deste tipo � diminuir os erros causados pelos sonares, dispositivos imprecisos. Contudo, resultados experimentais sugerem que esta abordagem talvez n�o seja a mais eficaz, como descrito na se��o \ref{sec:criticasbarra}.

\subsubsection{Vis�o Geral}
De modo geral, a opera��o deste modelo pode ser dividida em quatro etapas. Primeiramente, as �ltimas $k$ observa��es s�o validadas e � determinado se uma parede est� sendo vista pelo sonar. Se isso for verdade, consulta-se o mapa do ambiente de modo a se associar as leituras mais recentes a uma parede no mapa. Por fim, quando a associa��o � realizada com sucesso, determina-se a observa��o que seria esperada para a postura estimada do rob�. A associa��o � registrada e o modelo tenta mant�-la nas pr�ximas itera��es, atribuindo uma pontua��o para seu grau de confian�a.

\paragraph{Valida��o das observa��es} As �ltimas $k$ observa��es s�o armazenadas, juntamente com a postura estimada em cada uma delas. Determina-se, atrav�s de um teste estat�stico, se as observa��es podem corresponder a uma superf�cie plana. Tamb�m nesta fase, curvas realizadas pelo rob� s�o detectadas. Nesse caso, o \textit{buffer} � limpo e a associa��o atual, se houver, � desfeita.

\paragraph{Associa��o das observa��es com o mapa} O mapa � ent�o consultado para encontrar uma parede que corresponda �s observa��es validadas. O processo � bastante conservador. Se o algoritmo de correspond�ncia encontrar uma, e somente uma, parede que corresponda �s leituras do sonar, a associa��o � realizada com sucesso.

\paragraph{Rastreamento de associa��es} Caso uma associa��o seja realizada com sucesso, d�-se in�cio ao processo de rastreamento dessa associa��o. Para as pr�ximas leituras realizadas pelo sonar, a etapa de associa��o das observa��es � pulado - ou seja, mapa n�o � consultado e assume-se que a parede observada � aquela sendo rastreada. Estabelece-se uma pontua��o para cada rastreamento, em fun��o do erro entre as observa��es esperadas e obtidas. Quando a pontua��o se torna negativa, o rastreamento se encerra.

\paragraph{Obten��o das observa��es esperadas} De posse da parede que o sonar est� enxergando e da postura estimada do rob�, retorna-se a leitura esperada do sonar.

\subsection{Valida��o das Observa��es}
Definimos $k$ como o n�mero de observa��es coletadas pelo sonar, e $k_{min}$ como o n�mero m�nimo de observa��es para que seja poss�vel a valida��o de uma superf�cie plana. Enquanto $k<k_{min}$ a valida��o falhar�. Quando $k=k_{min}$, inicia-se o processo de valida��o.

Para determinar se as �tlimas \textit{k} observa��es correspondem a uma parede, calcula-se a rela��o entre a diferen�a entre observa��es sucessivas e a dist�ncia percorrida pelo rob� entre elas. A figura \ref{fig:modelo_sonar} ilustra que, caso um sonar esteja observando uma superf�cie plana, as leituras do sensor se alinhar�o em um segmento de reta. Isso equivale a dizer que a rela��o entre as diferen�as de leituras consecutivas e a dist�ncia percorrida entre essas leituras � constante. Matematicamente, seja $z_{i,j} = z^{(i)} - z^{(j)}$ a diferen�a entre duas medidas sucessivas e $d_{i,j}$ a dist�ncia percorrida entre elas, ent�o:

\begin{equation}
\frac{z_{2,1}}{d_{2,1}} = \frac{z_{3,2}}{d_{3,2}} = \dots = \frac{z_{i,j}}{d_{i,j}} = \gamma
\end{equation}

Idealmente, a rela��o � perfeita e $\frac{z_{i,j}}{d_{i,j}} = \gamma$ para quaisquer $i$ e $j$. Naturalmente, erros de medida introduzem varia��es nas observa��es, de modo que:

\begin{equation}
\frac{z_{2,1}}{d_{2,1}} = \gamma_{2,1} \approx \frac{z_{3,2}}{d_{3,2}} = \gamma_{3,2} \approx \dots \approx \gamma_{i, j}
\end{equation}

\begin{figure}
%	\centering
		\includegraphics{imagens/sonar/modelo_sonar.pdf}
		\caption{Representa��o das observa��es de um sonar com �ngulo relativo $\theta'_{sonar}$. Os feixes do sonar incidem na parede com um �ngulo $\beta$. Cada $z^{(i)}$ � uma observa��o, e $d_{i,j}$ � a dist�ncia percorrida entre as observa��es $i$ e $j$. Como a parede � uma superf�cie plana, a rela��o entre a diferen�a das leituras e a dist�ncia percorrida � constante.}
	\label{fig:modelo_sonar}
\end{figure}

Assumindo-se que um �nico segmento de reta est� sendo visto em todas as $k$ observa��es, podemos considerar que todos os elementos do conjunto formado por $\gamma$ s�o elementos de uma mesma distribui��o normal. Se isso for verdade, as observa��es s�o validadas e passa-se � pr�xima fase do modelo de observa��o. O seguinte teste de hip�tese pode ser utilizado para averiguar se o conjunto de $\gamma_{i,j}$ corresponde a uma distribui��o:

\begin{equation*}
		H_{0}:\gamma ~ N(A, \sigma^{2}_{obsmedia})
\end{equation*}
\begin{equation*}
		H_{1}:\gamma ~ N(A, \sigma^{2}_{alternativo}), \sigma^{2}_{obsmedia} < \sigma^{2}_{alternativo}
\end{equation*}
sendo que o teste � realizado sobre a estat�stica $\chi^{2}$.
Para aceitar $H_{0}$, � verificado se $\chi^{2}$ � menor que um dado limite:
\begin{equation}
	\chi^{2}_{k-2} < \chi^{2}_{k-2,\alpha}
	\label{eq:chitest}
\end{equation}
onde o segundo termo � tabelado, $\alpha$ indica a confian�a no teste e o primeiro termo � obtido da seguinte equa��o:
\begin{equation*}
	\chi^{2}_{k-2} = \frac{(k-2)s^{2}_{\gamma}}{\sigma^{2}_{obsmedia}}
\end{equation*}
onde $s^{2}_{\gamma}$ � a vari�ncia do conjunto formado pelos $\gamma_{i,i-1}$ de cada par de observa��es.

Considerando que todos os $\gamma_{i,i-1}$ pertencem � mesma distribui��o de probabilidades, $\sigma^{2}_{obsmedia}$ � a vari�ncia esperada dessa distribui��o. De acordo com \cite{barra}, seu valor � dado por:

\begin{equation}
\sigma^2_{obsmedia}=\sigma^2_{\gamma} = \vec{F}_\gamma \bm{\Sigma}_{\omega_{i,i-1}} \vec{F}_\gamma^{T} 
\end{equation}

\begin{align}
\vec{F}_\gamma=\frac{\partial\gamma(\omega_{i,i-1})}{\partial\omega_{i,i-1}}  |  \bm{\omega_{i,i-1}} = \bm{\bar{\omega}_{i,i-1}} = \notag \\
= \left( -\frac{(k-1)z_{i,i-1}}{d^{2}_{k,1}}~\frac{k-1}{d_{k,1}} \right) | \bm{\omega_{i,i-1}}
\end{align}

\begin{equation}
\bm{\Sigma}_{\omega_{i,i-1}} =
\begin{pmatrix}
\frac{\sigma^2_{d_{k,1}}}{(k-1)^2} & E_z\left[ z_{i, i-1} \frac{d_{k,1}}{k-1} \right] \\
 E_z\left[ z_{i, i-1} \frac{d_{k,1}}{k-1}\right] & \sigma^2_{z_{i,i-1}}
\end{pmatrix}
\end{equation}

\begin{equation}
\sigma^2_{z_{i,i-1}}=2\sigma^2_{obs} + \frac{\sigma^2_{d_{k,1}}}{(k-1)^2} \sin^2(\alpha)
\end{equation}

\begin{equation}
 E_z\left[ z_{i, i-1} \frac{d_{k,1}}{k-1} \right] = \frac{\sigma^2_{d_{k,1}} \sin(\alpha)}{(k-1)^2}
\end{equation}

\begin{equation}
\bm{w_{i,i-1}}=
\begin{pmatrix}
\frac{d_{k,1}}{k-1} & z_{i,i-1}
\end{pmatrix}^T
\end{equation}
Nas equa��es acima, $d_{k,1}$ � a dist�ncia percorrida pelo rob� durante as �ltimas $k$ observa��es; $z_{i,j}$ � a diferen�a entre a leitura dos sonares nos instantes $i$ e $j$; $\sigma^2_{obs}$ � a incerteza da observa��o do sonar, determinada empiricamente como 625 $mm^2$; $\sigma^2_{d_{k,1}}$ � a incerteza no deslocamento do rob� e vale $\frac{d_{k,1}\sigma_{desloc}}{k-1}$, com $\sigma_{desloc}$ determinado experimentalmente como 0.05 m; e $\alpha$ � o �ngulo de incid�ncia do rob� com a parede observada. $\alpha$ pode ser calculado empiricamente do seguinte modo:

\begin{equation}
\sin \alpha=\frac{z_{k,1}\sin \theta'_{sonar}}{\sqrt{z_{k,1}^2 + d_{k,1}^2 - 2z_{k,1} d_{k,1} \cos \theta'_{sonar}}}
\label{eq:novoalpha}
\end{equation}

\subsection{Associa��o das observa��es com o mapa}

Se as $k$ observa��es armazenadas passarem no teste estat�stico da equa��o \ref{eq:chitest}, procede-se com a busca de uma parede mapeada que corresponda �s leituras validadas. O algoritmo utilizado trata as paredes como retas infinitas, definidas em termos de $r_{wall}$, o comprimento do segmento perpendicular � reta e que passa pela origem, e $\theta_{wall}$, o �ngulo deste segmento com rela��o � origem. A figura \ref{fig:retas} ilustra essa conven��o.

Dois filtros s�o aplicados sobre as paredes do mapa, de modo a reduzir o espa�o de busca. Primeiramente, paredes que estejam fora do alcance do sonar s�o eliminadas; em seguida, paredes que estejam al�m do cone de vis�o do sonar tamb�m s�o desconsideradas.

\subsubsection{Filtro de paredes distantes}

Para cada parede do mapa, representada em termos de $r_{wall}$ e $\theta_{wall}$, calcula-se a dist�ncia perpendicular at� a �ltima posi��o estimada do rob�, $\hat{x}$. Caso essa dist�ncia seja maior do que um dado limite -- a faixa de opera��o do sonar utilizado --, rejeita-se esta parede como candidata a associa��o. Empiricamente, determinou-se 400 cm como um limite razo�vel.

\subsubsection{Filtro de paredes pelo �ngulo do sonar}

� necess�rio tamb�m filtrar as paredes candidatas pelo �ngulo global do sonar. Neste caso, as paredes n�o s�o tratadas como retas infinitas, e sim como segmentos. Um segmento s� pode ser observado se um cone centrado na postura global do sonar com �ngulo de abertura $\phi$ interseccion�-lo. O algoritmo \ref{algo:cone} descreve o comportamento do filtro.

\begin{figure}[bt]
	\centering
	\includegraphics[width=.5\columnwidth]{imagens/retas.pdf}
	\caption{O modelo adotado para representar as paredes do mapa, que s�o tratadas como retas infinitas. Cada reta � definida em termos de $r_{wall}$ e $\theta_{wall}$.}
	\label{fig:retas}
\end{figure}

\begin{algorithm}
	\dontprintsemicolon
	\SetKwData{bordau}{borda1}
	\SetKwData{bordad}{borda2}
	\SetKwData{eixo}{eixo}
	\SetKwData{parede}{parede}
	\SetKwData{pareta}{retaParede}
	\SetKwData{verdade}{verdadeiro}
	\SetKw{e}{e}
	\SetKw{lp}{$($}
	\SetKw{rp}{$)$}
	\SetKw{ou}{ou}
	\SetKwData{falso}{falso}
	\SetKwData{interbu}{interse��oBorda1}
	\SetKwData{interbd}{interse��oBorda1}
	
	\Entrada{A postura global estimada do sonar, $(\hat{x}_{sonar}, \hat{y}_{sonar}, \hat{\theta}_{sonar})$, e seu �ngulo de abertura, $\phi$}
	\Entrada{A parede, representada por um segmento de reta (\parede)}
	\Saida{\verdade se a parede for vis�vel pelo sonar; \falso caso contr�rio}
	
	\Inicio{
		\pareta $\leftarrow$ a reta que cont�m \parede \;
		\bordau $\leftarrow$ a semireta com origem em $(\hat{x}_{sonar}, \hat{y}_{sonar})$ e inclinina��o $\hat{\theta}_{sonar} + \phi/2$ \;
		\bordad $\leftarrow$ a semireta com origem em $(\hat{x}_{sonar}, \hat{y}_{sonar})$ e inclinina��o $\hat{\theta}_{sonar} - \phi/2$ \;
		\eixo $\leftarrow$ a semireta  com origem em $(\hat{x}_{sonar}, \hat{y}_{sonar})$ e inclinina��o $\hat{\theta}_{sonar}$ \;
		\interbu $\leftarrow$ a interse��o, se existir, entre \bordau e \pareta \;
		\interbd $\leftarrow$ a interse��o, se existir, entre \bordad e \pareta \;
		\Se{\lp \interbu  \ou \interbd\ \rp \e \parede cont�m o ponto de interse��o}{\Retorna{\verdade} \;}
		\SenaoSe{\interbu \e \interbd}{
			\SetKwData{seginter}{segmentoInterse��es}
			\seginter $\leftarrow$ o segmento de reta que une as duas interse��es \;
			\lSe{ \seginter cont�m \parede }{\Retorna{\verdade}} \;
			\lSenao{\Retorna{\falso}}
		}
		\SenaoSe{ \interbu \ou \interbd } {
			\SetKwData{angpu}{anguloExtremo1}
			\SetKwData{angpd}{anguloExtremo2}
			\angpu $\leftarrow$ $atan2($\parede.extremo1.y, \parede.extremo1.x$)$ \;
			\angpd $\leftarrow$ $atan2($\parede.extremo2.y, \parede.extremo2.x$)$ \;
			\lSe{
			$distanciaAngular($ \eixo, \angpu $) \leq \phi / 2$ \ou 
			$distanciaAngular($ \eixo, \angpd $) \leq \phi / 2$}{\Retorna{\verdade} \;
			} \;	
		}
		\Senao{ \Retorna{\falso}}
			
	
	}
	
	\label{algo:cone}
	\caption{Algoritmo de filtro de uma parede pelo �ngulo do sonar}
\end{algorithm}


%\end{document}
\subsection{Modelo de Observa��o da Vis�o}\label{subsec:modeloVisao}

O uso de vis�o � muito interessante para a obten��o da localiza��o pois uma c�mera de v�deo � capaz de captar uma grande quantidade de est�mulos visuais. Dentre os v�rios tipos de est�mulos, optou-se, baseado em \cite{barra}, como principal est�mulo retas verticais presentes nos ambientes, tais como batentes de portas, colunas, janelas, etc. A denomina��o \textsl{reta vertical} ser� dada �s retas presentes no mundo real, enquanto \textsl{proje��es} s�o as observa��es dessas retas nas imagens.

Neste trabalho, emprega-se a vis�o monocular, cujo sensor � uma c�mera de v�deo montada sobre o rob�, apontada na dire��o frontal dele. 

\subsubsection{Modelo da C�mera}
Dada a caracter�stica de baixo custo desejada, a c�mera usada � uma \textit{webcam} convencional, baseada na tecnologia CCD. O modelo da c�mera segui o modelo de c�mera de orif�cio, tamb�m conhecido como \textit{pinhole}. Neste modelo, a c�mera possui um centro focal C e um plano de proje��o, onde as imagens s�o formadas, representado como um espa�o discreto cujo tamanho indica a resolu��o da c�mera e cada elemento discreto � um pixel da imagem. 

\begin{figure}[htbp]
	\centering
		\includegraphics{imagens/pinhole.jpg}
	\caption{Modelo de c�mera pinhole, composto pelo centro focal C e pelo plano de proje��o (figura retirada de \cite{barra})}
	\label{fig:pinhole}
\end{figure}


As cores s�o representadas segundo o modelo RGB, em que uma cor � descrita por meio das intensidades nos componentes vermelho, verde e azul do espectro de luz. 


Considera-se o sistema de coordenadas $\left( X_c, Y_c, Z_c \right)$ intr�nseco � c�mera. Um ponto qualquer do espa�o pode ser representado nesse sistema como ${\bf p_{cam}} = \left[ x_{cam}, y_{cam}, z_{cam} \right]^T$, enquanto sua proje��o no plano de proje��o da c�mera � dada por ${\bf p_{proj}} = \left[ u, v \right]^T$. As coordenadas de ${\bf p_{proj}}$ podem ser obtidas segundo as equa��es abaixo:

\begin{equation}
	u = f_u \frac{y_{cam}}{x_{cam}} + u_0
	\label{coordu:pinhole}
\end{equation}

\begin{equation}
	v = f_v \frac{z_{cam}}{x_{cam}} + v_0
	\label{coordv:pinhole}
\end{equation}

onde $f_u$ e $f_v$ indicam a dist�ncia focal da c�mera horizontal e vertical, respectivamente, cujos valores s�o obtidos em uma etapa de calibra��o da c�mera, assim como os valores de $u_0$ e $v_0$, que indicam o centro do plano de proje��o, no sistema de coordenadas do plano de proje��o $\left(U, V\right)$. Os valores de $u$ e $v$ s�o dados em pixels. 

J� um ponto no sistema de coordenadas global, isto �, no mesmo sistema de coordenadas na qual a postura do rob� � representada, dado por ${\bf p_real} = \left[x, y, z\right]^T$, precisa ser previamente convertido para o sistema de coordenadas intr�nseco � c�mera, resultando em um ponto ${\bf p_cam} = \left[ x_{cam}, y_{cam}, z_{cam} \right]^T$, antes que as equa��es acima possam ser aplicadas. Tal convers�o � feita por meio de:

\begin{equation}
	{\bf p_{cam}} = {\bf R^T\left( p_{real} - C \right)}
\end{equation}
	
onde ${\bf R}$ � o par�metro de rota��o relativa entre o sistema de coordenadas da c�mera e o global, enquanto ${\bf C}$ � a transla��o entre esses sistemas de coordenadas. Assim como $f_u$ e $f_v$, ${\bf R}$ e ${\bf C}$ tamb�m s�o determinados durante a calibra��o da c�mera.
	

\subsubsection{Realce de Proje��es Verticais}

A primeira etapa do processamento da vis�o � ressaltar, nos quadros obtidos pela c�mera, apenas os pixels que possam pertencer � proje��es de retas verticais. Isso � feito atrav�s do uso de um Operador de Sobel modificado.

O Operador de Sobel � um algoritmo voltado para a detec��o de bordas em imagens, cujo resultado � uma imagem em tons de cinza onde os pixels que representam bordas de objetos ficam demarcados em branco enquanto os demais assumem a tons pr�ximos ao preto. 

Vale ressaltar que o Operador de Sobel � um tipo de convolu��o de imagens. A ideia b�sica da convolu��o de imagens, que � discreta e bidimensional, � a de uma janela que � deslizada sobre a imagem. O valor do pixel resultante � igual � soma ponderada dos pixels da imagem original que se encontram dentro da janela. Os pesos s�o os valores do filtro que foram estabelecidos para cada um dos pixels da janela. Tal janela � denominada \textit{kernel} da convolu��o. O Operador de Sobel permite que seu filtro seja divido em dois, sendo um para a dire��o horizontal e o outro, para a vertical.

Uma vez que o est�mulo de interesse s�o as proje��es verticais, utiliza-se apenas o filtro horizontal do Operador de Sobel. Como dito anteriormente, foi utilizado uma vers�o modificada do filtro, cujos valores foram determinados empiricamente considerando-se dois par�metros: precis�o na demarca��o das retas verticais e desempenho, uma vez que aplicar a convolu��o sobre uma imagem � um processo computacionalmente pesado. Assim, o filtro obtido apresenta a seguinte matriz:

\begin{equation}
	{\bf filtro_{vertical}} = \left(
		\begin{matrix}
			2 & 0 & -2 \\
	        2 & 0 & -2 \\
	        2 & 0 & -2 \\
	        2 & 0 & -2 \\
	        2 & 0 & -2 \\
	        2 & 0 & -2 \\
	        2 & 0 & -2
		\end{matrix}
	\right)
\end{equation}

� importante salientar que antes do Operador de Sobel ser aplicado, a imagem sofre uma opera��o de borramento, cujo objetivo � reduzir a influ�ncia que o ru�do presente na captura das imagens possa ter sobre a qualidade do resultado do Operador de Sobel.

\begin{figure}[ht]
	\centering
		\includegraphics[width=.6\columnwidth]{imagens/corredor.jpg}
	\caption{Foto tirada no corredor das salas C2}%
	\label{visao:original}
\end{figure}

\begin{figure}[ht]
	\centering
		\includegraphics[width=.6\columnwidth]{imagens/sobel.jpg}
	\caption{Imagem obtida ap�s a aplica��o do Operador de Sobel modificado sobre a Figura~\ref{visao:original}}
	\label{visao:convolucao}
\end{figure}


\subsubsection{Perfil de Cor}

O perfil de cor segue a hip�tese que a distribui��o de cores ao redor das retas verticais cont�m informa��o suficiente para ajudar a identificar proje��es de uma mesma reta vertical em diferentes quadros do v�deo.

Ele considera uma regi�o da imagem centrada na sequ�ncia de pixels que pertencem a uma proje��o, estendendo-se por uma faixa de \textit{n} pixels a direita e a esquerda de cada pixel da proje��o.

As informa��es que se mostraram relevantes sobre o perfil de cor s�o os valores m�dios, para cada componente de cor, denominados $R_{esq}$, $R_{dir}$, $G_{esq}$, $G_{dir}$, $B_{esq}$ e $B_{dir}$ dos lados esquerdo e direito em rela��o ao centro da regi�o.

A compara��o de dois perfis de cor resulta num fator de correla��o normalizado que indica qu�o semelhantes s�o os perfis. O c�lculo desse fator segue as seguintes equa��es:

\begin{eqnarray*}
R & = & \left( \delta - \vert R^{(1)}_{esq} - R^{(2)}_{esq} \vert \right) + \left( \delta - \vert R^{(1)}_{dir} - R^{(2)}_{dir} \vert \right) \\
G & = & \left( \delta - \vert G^{(1)}_{esq} - G^{(2)}_{esq} \vert \right) + \left( \delta - \vert G^{(1)}_{dir} - G^{(2)}_{dir} \vert \right) \\
B & = & \left( \delta - \vert B^{(1)}_{esq} - B^{(2)}_{esq} \vert \right) + \left( \delta - \vert B^{(1)}_{dir} - B^{(2)}_{dir} \vert \right)
\end{eqnarray*}
\begin{displaymath}
C = \frac{\left( R + G + B \right)}{6\delta} 
\label{eqn:perfilcor}
\end{displaymath}

onde $\delta$ � a maior diferen�a esperada entre os componentes, $R$, $G$ e $B$ s�o representam as diferen�as nas intensidades dos componentes de cor entre o lado direito e esquerdo, em rela��o � reta considerada, e $C$ � o fator de correla��o. 


\subsubsection{Detec��o de Proje��es Verticais}

A partir da imagem resultante da etapa anterior, onde existem apenas linhas verticais demarcada, as proje��es s�o detectadas e suas posi��es na imagem s�o determinadas, juntamente com outros par�metros que s�o citados abaixo.

Antes de iniciar a detec��o, aplica-se um procedimento que filtra pixels que n�o s�o m�ximos locais, isto �, dada uma pequena por��o da imagem, a intensidade de um pixel n�o � a maior dessa regi�o. J� que a tend�ncia da etapa de extra��o � marcar os pixels mais pr�ximos do centro de uma linha vertical com valores maiores que os pixels perif�ricos da mesma linha, o procedimento aplicado faz com que uma poss�vel proje��o, antes representada na imagem como uma linha relativamente grossa (largura de v�rios pixels), passe a ter largura de um �nico pixel.

O procedimento de detec��o � iniciado em um estado no qual a imagem � varrida, da esquerda para a direita, de cima para baixo, buscando um pixel que tenha intensidade maior do que um \textsl{limiar de in�cio} ($lim_{inicio}$). Quando um pixel atende a esse requisito, passa-se a um estado secund�rio.

Neste novo estado, sup�e-se que o pixel acima do limiar � um ponto pertencente a um nova proje��o, adicionando-o a uma lista de pontos que comp�em a proje��o. A partir dele, a imagem passa a ser varrida na dire��o vertical, seguindo as seguintes regras: compara-se o valor dos tr�s pixels vizinhos que est�o diretamente abaixo do pixel rec�m-adicionado na nova poss�vel proje��o. O pixel que ser� adicionado na lista da proje��o ser� aquele com o maior valor. Contudo, existem algumas ressalvas:
\begin{itemize}
\item se o pixel de maior valor for um dos inferiores laterais, entra em a��o um fator de in�rcia vertical cujo papel � for�ar a proje��o a ser o mais vertical poss�vel. Enquanto este fator estiver ativo, � dada prefer�ncia para o pixel diretamente abaixo, mesmo que seu valor n�o seja o maior. Esse fator ser� desativado quando alguns pixels n�o m�ximos forem selecionados, evitando, assim, a forma��o de um proje��o distorcida.
\item se o maior valor for inferior a um \textsl{limiar de t�rmino} ($lim_{termino}$), o pixel � adicionado temporariamente � proje��o e passa-se a contar quantos pixels abaixo do limiar foram inseridos em sequ�ncia. Caso essa contagem ultrapasse um valor m�ximo, todos os que foram adicionados temporariamente s�o removidos, a proje��o obtida at� ent�o � considerada finalizada e o algoritmo de detec��o volta ao estado inicial. Se ap�s alguns pixels abaixo do limiar de t�rmino serem inseridos, mas com contagem menor do que a m�xima permitida, for encontrado um pixel cuja intensidade � superior ao limiar de t�rmino, a adi��o deles � confirmada e a contagem � zerada.
\end{itemize}

Cada proje��o detectada passa por algumas verifica��es, que incluem o tamanho da proje��o, eliminando proje��es muito pequenas, e a inclina��o, descartando aquelas que n�o est�o t�o verticais quanto desejado. Aquelas que forem aceitas ter�o seus perfis de cor determinados, o que conclui a fase de detec��o.

Ao final, cada proje��o ser� representada a partir da posi��o no eixo horizontal do plano de proje��o da c�mera (coordenada $u_vert$) e seu respectivo perfil de cor.

\begin{figure}[ht]
	\centering
		\includegraphics[width=.6\columnwidth]{imagens/projections.jpg}
	\caption{Proje��es detectadas, assinalada em vermelho, a partir da imagem apresentada na Figura ~\ref{visao:original}}
	\label{visao:projecoes}
\end{figure}

\subsubsection{Descri��o de Marcos}

Um marco pode ser definido como uma reta vertical cuja posi��o no mundo real � conhecida. Sua representa��o inclui as coordenadas de sua posi��o ${\bf p_{marco}} = \left[x, y\right]$, no sistema de coordenadas absolutas (globais), e seu perfil de cor, que pode ser usado como forma de identifica��o.

Os marcos s�o usados, primeiramente, para que se tente estabelecer um v�nculo entre proje��es que est�o sendo observadas com retas verticais do ambiente. Aqueles com os quais foi poss�vel realizar uma associa��o s�o usados para atualizar a estimativa da postura do rob�, conforme ser� mostrado posteriormente.


\subsubsection{Associa��o entre Proje��es Verticais e Marcos}

A fase de associa��o visa determinar quais marcos conhecidos do ambiente est�o sendo observados em um dado momento, a partir da lista de proje��es detectadas no quadro da c�mera. Ela pode ser segregada em tr�s passos, conforme descrito a seguir.

\paragraph{Restri��o de marcos}
Para reduzir o conjunto de marcos com os quais tentar-se-� estabelecer novas associa��es, e com isso, diminuindo a chance de ocorrerem falsos positivos, o conjunto de todos os marcos conhecidos do ambiente passa por um filtro que, usando a �ltima postura conhecida do rob�, calcula quais marcos est�o dentro do campo de vis�o da c�mera. 

Tendo-se determinado o �ngulo $alfa_{cam}$ relativo a abertura do campo de vis�o da c�mera, na horizontal, tra�a-se um setor circular de raio infinito, centrado nas coordenadas $\left[x, y\right]^T$ da postura do rob�, apontando para a mesma dire��o indicada por $\theta$, com �ngulo do setor igual a $alfa_{cam}$. Em seguida, � verificado se o ponto ${\bf p_{marco}} = \left[x_{marco}, y_{marco}\right]^T$ de cada marco encontra-se dentro desse setor circular. 

� importante observar que o filtro n�o remove marcos que possam estar oclu�dos, j� que o filtro n�o tem informa��es suficientes para tal procedimento, nem leva em considera��o a dist�ncia entre o rob� e o marco como um par�metro de restri��o, pois, apesar da probabilidade de uma associa��o ser concretizada diminui conforme essa dist�ncia aumenta, ela n�o pode ser completamente descartada neste passo inicial.


\paragraph{Janela de Busca por Previs�o}
Novamente, a partir da �ltima postura conhecida do rob�, � poss�vel prever a coordenada $u_{previsao}$ que a proje��o vertical de um marco deveria estar localizada na imagem, usando a equa��o \ref{coordu:pinhole} do modelo de c�mera de orif�cio. 

Para cada marco vis�vel $i$, h� uma janela de busca, centrada em $u^(i)_{previsao}$, com largura igual a $larg_{janela}$, em que apenas as proje��es contidas nesse janela t�m chance de serem associadas ao marco dono da janela. Portanto, a janela de busca atua como uma restri��o por disposi��o espacial, reduzindo o n�mero de compara��es que precisar�o ser feitas at� a conclus�o da associa��o.


\paragraph{Determina��o da associa��o por Perfil de Cor}
As proje��es que sobraram ter�o seus perfis de cor comparados com o do marco, resultando em valores que indicam o n�vel de correla��o entre uma proje��o e o marco. Tal compara��o faz uso da equa��o \ref{eqn:perfilcor}. Uma associa��o entre proje��o vertical e marco ser� concretizada se existir pelo menos um valor de correla��o maior do que 0 e a proje��o associada ao marco ser� aquela que apresentar o maior valor de correla��o.


\subsubsection{Obten��o da observa��o esperada}

Ap�s a fase de associa��es, tem-se a lista de marcos que est�o sendo observados e que devem ser usados para atualizar a postura do rob�. O pr�ximo passo � fornecer ao filtro de Kalman estendido as informa��es de observa��o real obtida ${\bf z}$, os valores esperados para as observa��es ${\bf h}$, a matriz de covari�ncia $\bm{R}$, e o jacobiano de ${\bf h}$, a matriz $\bm{H}$.

\paragraph{C�lculo da observa��o esperada}

Seja a postura do rob� descrita como ${\bf {x}} = \left[ x, y, \theta\right]^T$ e ${\bf p^{(i)}_{marco}} = \left[ x^{(i)}_{marco}, y^{(i)}_{marco}\right]^T$ as coordenadas de cada marco \textit{i} observado, tem-se:

\begin{equation*}
h^{(i)}({\bf x}) = -f_u \frac{ \sin(\theta)\left(x^{(i)}_{marco} - x \right) + \cos( \theta) \left( y^{(i)}_{marco} - y \right) } { \cos(\theta) \left(x^{(i)}_{marco} - x \right) - \sin(\theta) \left( y^{(i)}_{marco} - y \right) } + u_0 
\end{equation*}

onde $f_u$ � a dist�ncia focal na dire��o horizontal enquanto $u_0$ � o centro horizontal do plano de proje��o da c�mera. Maiores detalhes sobre a determina��o da equa��o $h\left( {\bf x} \right)$ podem ser obtidos em \cite{barra}.

O vetor ${\bf x}$ � composto pelo resultado da equa��o acima, aplicada sobre cada um dos marcos observados, tendo dimens�o $1Xi$. 


\paragraph{C�lculo do jacobiano} A matriz $\bm{H}$ � necess�ria para que o EKF corrija a postura corretamente em cada componente. Ela � dada por:

\begin{equation*}
\bm{H} = \left. \frac{\partial h(\vec{x})}{\partial \vec{x}}\right|_{\hat{x}}
\end{equation*}

\begin{equation}
\bm{H} = 
	\left(
    	\begin{array}{>{\displaystyle}c}
    		f_u \frac{ \sin(\theta)Z - \cos(\theta)V}{Z^2} \\
    		\\
    		f_u \frac{ \cos(\theta)Z + \sin(\theta)V}{Z^2} \\
    		\\
    		f_u \left[ -\frac{V^2}{Z^2} + 1 \right]
		\end{array}
	\right)^T
\end{equation}

onde $V$ e $Z$ s�o vari�veis auxiliares dadas por:

\begin{equation*}
	V = \sin(\theta)\left(x^{(i)}_{marco} - x \right) + \cos( \theta) \left( y^{(i)}_{marco} - y \right)
\end{equation*}

\begin{equation*}
	Z = \cos(\theta) \left(x^{(i)}_{marco} - x \right) - \sin(\theta) \left( y^{(i)}_{marco} - y \right)
\end{equation*}

Cada linha da matriz $\bm{H}$ diz respeito a um marco observado, devendo ter o n�mero de linhas igual ao n�mero de marcos. 

\paragraph{C�lculo da matriz de covari�ncia} A matriz $\bm{R}$ � uma matriz quadrada que cont�m as covari�ncias que s�o passadas para o EKF durante a etapa de atualiza��o da estimativa. Sendo $n$ o n�mero de marcos observadas em um determinado ciclo da vis�o, a matriz ter� dimens�o $nXn$, com os seguintes valores:

\begin{equation}
	R = 
	\left(
	\begin{matrix}
		\sigma^2_{obs} & 0 & 0 & \ldots \\
		0 & \sigma^2_{obs} & 0 & \ldots \\
		0 & 0 & \sigma^2_{obs} & \ldots \\
		\vdots & \vdots & \vdots & \ddots \\
	\end{matrix}
	\right)
\end{equation}

onde $\sigma^2_{obs}$ � a vari�ncia medida empiricamente.