Um componente essencial no sistema de um rob� m�vel � a habilidade de se localizar com acur�cia, o que envolve estimar sua postura em rela��o a uma representa��o global do ambiente em que se encontra. A especifica��o de uma abordagem de localiza��o baseada em dados sensoriais possui uma estimativa inicial da postura do rob�s usa os dados coletados pelos sensores, juntamente com os dados que podem ser obtidos de uma mapa do ambiente, para produzir uma estimativa precisa da postura. Uma dificuldade � que os dados sensoriais s�o corrompidos por erros de medidas, decorrentes, principalmente, de ru�do. O uso de dados originados em m�ltiplos sensores fornece informa��o redundante e complementar, que pode ser utilizada para gerar uma estimativa combinada, aumentando a confian�a na postura calculada. Este trabalho versa sobre o desenvolvimento de um rob� m�vel de baixo custo, equipado com od�metro, um semi-anel contendo oito sonares e uma c�mera de v�deo convencional, capaz de guiar pessoas a regi�es espec�ficas de um ambiente, de acordo com o destino que lhe for indicado. As estimativas de postura s�o calculadas a partir de um Filtro de Kalman Estendido (EKF).