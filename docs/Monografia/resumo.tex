Neste projeto, � desenvolvido um rob�-guia para ambientes internos chamado SAURON. O sistema � composto por dois m�dulos, localiza��o e navega��o. Localiza��o significa estimar a postura do rob� em rela��o a uma representa��o global do ambiente em que se encontra. A especifica��o de uma abordagem de localiza��o baseada em dados sensoriais possui uma estimativa inicial da postura do rob� e usa os dados coletados pelos sensores, juntamente com os dados que podem ser obtidos de uma mapa do ambiente, para produzir uma estimativa precisa da postura. Uma dificuldade � que os dados sensoriais s�o corrompidos por erros de medidas, decorrentes, principalmente, de ru�do. O uso de dados originados em m�ltiplos sensores fornece informa��o redundante e complementar, que pode ser utilizada para gerar uma estimativa combinada, aumentando a confian�a na postura calculada. Este trabalho versa sobre o desenvolvimento de um rob� m�vel de baixo custo, equipado com od�metro, um semi-anel contendo oito sonares e uma c�mera de v�deo convencional, capaz de guiar pessoas a regi�es espec�ficas de um ambiente, de acordo com o destino que lhe for indicado. As estimativas de postura s�o calculadas a partir de um Filtro de Kalman Estendido (EKF). O SAURON � capaz de navegar intermapas, de modo que possa percorrer mais de um andar de um edif�cio. A busca de rota intermapa � realizada atrav�s de uma busca em profundidade, ao passo o A* � utilizado para rotas intramapa. Os resultados obtidos foram satisfat�rios. O rob� foi capaz de atingir seus destinos com erros abaixo de 50 cm, valor aceit�vel para os prop�sitos do projeto.