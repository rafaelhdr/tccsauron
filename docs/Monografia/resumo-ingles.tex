\selectlanguage{english}

A key component of an autonomous mobile robot system is the ability to locate itself accurately, which involves estimating its pose on a global representation of the environment. The specification of a sensor-based localization approach has an initial estimate of the robot's pose and uses sensor data, along with data that can be obtained from the environment's map, to produce an accurate estimate of the pose. One of the main difficulties is that sensor data is corrupted by measurement errors, mostly caused by noise. The use of data from multiple sensors provides redundant and complementary information that can be used to generate a combined estimate aiming at an increase in the confidence of the final pose estimate. This work focused on developing a low cost mobile robot, equipped with odometers, a semi-ring of eight sonars and a conventional video camera, able to guide people to specific regions of an environment, according to the destination that is indicated. Estimates of posture are calculated using an Extendend Kalman Filter (EKF).