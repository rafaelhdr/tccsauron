\chapter{Conclus�es} \label{chp:Conclusoes}


O modelo simples de observa��o dos sonares surgiu como uma alternativa ao modelo original, que buscava validar as medidas para que somente as mais, apresentando caracter�sticas, como a elevada frequ�ncia com que novas medidas s�o fornecidas, que permitem aproveitar o principal aspecto do filtro de Kalman, que � a capacidade de corre��o de estimativas mesmo com medidas de baixa qualidade dos sensores.

A arquitetura de software desenvolvida apresentou uma grande efici�ncia, sendo capaz de realizar todas as tarefas necess�rias (execu��o da navega��o, calculo de estimativas, an�lise dos quadros da c�mera), sem consumir toda a capacidade de processamento do computador embarcado no rob�, e com isso, possibilitando que a taxa de obten��o de medidas dos sensores pudesse ser alta, ficando limitada apenas pela capacidade f�sica deles.
