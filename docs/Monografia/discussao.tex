\chapter{Discuss�o}

Recomenda-se que seja uma objetiva considera��o dos resultados apresentados anteriormente e que conduza �s principais conclus�es.\\
Neste item o autor tem maior liberdade de express�o, o que coloca em evid�ncia a sua maturidade intelectual.\\
Na discuss�o dos resultados, o autor deve:
\begin{itemize}
	\item relacionar causas e efeitos;
	\item estabelecer, a partir dos experimentos, a dedu��o das generaliza��es e princ�pios b�sicos;
	\item elucidar contradi��es, teorias e princ�pios relativos ao trabalho;
	\item indicar a aplicabilidade dos resultados obtidos e suas limita��es;
	\item elaborar, se poss�vel, uma teoria para justificar os resultados obtidos
	\item sugerir novas pesquisas, a partir das experi�ncias adquiridas no desenrolar do trabalho, visando sua complementa��o.
\end{itemize}
N�o � aconselh�vel a jun��o dos resultados com a discuss�o, formando um �nico cap�tulo. Entretanto, se esta forma for adotada, os resultados devem ser discutidos na medida em que forem apresentados.