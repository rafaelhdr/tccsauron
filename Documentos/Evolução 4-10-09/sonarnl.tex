\documentclass[a4paper,11pt]{article}
\usepackage{graphicx}
\usepackage[brazilian]{babel}
\usepackage[latin1]{inputenc}
\usepackage[T1]{fontenc}
\usepackage{fullpage}

\begin{document}
\section{Testes com a biblioteca SonARNL}

A empresa MobileRobots, fabricante do rob� Pioneer P2DX utilizado no projeto, disponibiliza a seus clientes uma biblioteca chamada SonARNL, que prov� fun��es de localiza��o e navega��o rob�ticas. O c�digo-fonte da biblioteca � fechado, e alguns exemplos em C++ distribu�dos no pacote ensinam como utiliz�-la.

\subsection{Ambiente de testes}
Os testes foram realizados no pr�dio da Engenharia El�trica da Escola Polit�cnica, mais especificamente no corredor das salas C2. Esse local foi escolhido por representar desafios comuns a todo o ambiente, em especial, alta simetria entre as duas paredes do corredor.

A primeira etapa na realiza��o dos testes � o mapeamento do ambiente. Obtivemos uma planta, no formato AutoCAD, de todo o pr�dio, que havia sido constru�da por ex-alunos da Prof. Anna. Convertemos parte dessa planta para o formato entendido pela biblioteca SonARNL, e verificamos manualmente a corre��o das medidas. Localizamos algumas imprecis�es (por vezes grosseiras), que foram corrigidas. � medida que os testes progrediram, o mapa foi refinado cada vez mais.

\subsection{Resultado dos testes}
Testamos o rob� em duas ocasi�es. No primeiro dia, montamos um circuito assim�trico e pequeno, que ia da sala C2-50 ao corredor que termina nas escadas. O rob� foi capaz de se deslocar com sucesso no circuito, atingindo os objetivos selecionados. Para aferir a precis�o do rob�, marcamos 20 pontos no ambiente. O rob� foi instru�do a se dirigir a cada um deles e, quando a biblioteca informava ter atingido seu destino, med�amos o erro entre a posi��o real e a cren�a do rob�.

Foram realizadas tr�s baterias de testes. Nas duas primeiras, o rob� encontrou um ambiente perfeitamente est�tico e o erro m�dio foi de 30,18 cent�metros. No �ltimo teste, tentamos atrapalh�-lo, ficando na frente dos sonares. A diferen�a nos resultados foi percept�vel. 


\end{document}